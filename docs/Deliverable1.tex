\documentclass{article}
\usepackage[utf8]{inputenc}
\usepackage{listings}
\usepackage{amssymb,amsmath,amsthm}
\usepackage{xcolor}
 
\newcommand{\R}{\mathbb{R}}
\newcommand{\F}{\mathbb{F}}

\newcommand{\braces}[1]{\left(#1\right)}
\newcommand{\set}[1]{\left\{#1\right\}}
\newcommand{\Span}[1]{\text{Span}\left\{#1\right\}}

\newcommand{\vVectorXYZ}[3]{\begin{bmatrix}#1 \\ #2 \\#3\end{bmatrix}}

 
\theoremstyle{definition}
\newtheorem*{definition}{Definition}
\newtheorem*{convention}{Convention}
\newtheorem*{theorem}{Theorem}
\newtheorem*{example}{Example}
\newtheorem*{consider}{Consider}


\theoremstyle{remark}
\newtheorem*{remark}{Remark}
\newtheorem*{note}{Note}

\parskip 0.1in
 
\begin{document}
\section*{IMDb Movie Stereotyper}
Jacob Junqi Tian

\subsection*{Objective}
Predict the rating of a movie given information such as, 
\begin{itemize}
    \item Genre of the movie,
    \item Year of the movie, and
    \item A list of actors in the movie.
\end{itemize}

If time permits, the model could also judge the movie based on 
\begin{itemize}
    \item Its title,
    \item A (human-readable) summary of the plot, and
    \item The cover art of the movie (yes, we would judge the movie by its cover).

\end{itemize}

\subsection*{Technologies}
To approximate the rating of the movie based on its genre, year, and list of actor, regression would suffice. Given the overwhelming size of the dataset, \textbf{gradient descent} might be the best choice.

To approximate the rating based on its title or summary, it would take \textbf{natural language processing} techniques.

\textbf{Convolutional neural networks} might help judge the movie by its cover.
\end{document}